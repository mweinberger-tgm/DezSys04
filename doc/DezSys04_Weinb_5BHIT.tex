\documentclass[letterpaper, 12pt]{article}

%%%%%%%%%%%%%%%%%%%%%%%%%%%%%
% DEFINITIONS
% Change those informations
% If you need umlauts you have to escape them, e.g. for an ü you have to write \"u
\gdef\mytitle{Laborprotokoll}
\gdef\mythema{DezSys04}

\gdef\mysubject{Systemtechnik-Labor}
\gdef\mycourse{5BHIT 2015/16, Gruppe Z}
\gdef\myauthor{Michael Weinberger}

\gdef\myversion{1.0}
\gdef\mybegin{08. Januar 2016}
\gdef\myfinish{14. Januar 2016}

\gdef\mygrade{Note:}
\gdef\myteacher{Betreuer: Th.Micheler}
%
%%%%%%%%%%%%%%%%%%%%%%%%%%%%%

\input special/preamble.tex

\let\tempsection\section
\renewcommand\section[1]{\vspace{-0.3cm}\tempsection{#1}\vspace{-0.3cm}}
\WithSuffix\newcommand\section*[1]{\tempsection*{#1}}

\let\tempsubsection\subsection
\renewcommand\subsection[1]{\vspace{0cm}\tempsubsection{#1}\vspace{0cm}}

\let\tempsubsubsection\subsubsection
\renewcommand\subsubsection[1]{\vspace{0cm}\tempsubsubsection{#1}\vspace{0cm}}

\linespread{0.94}

\lhead{\mysubject}
\chead{}
\rhead{\bfseries\mythema}
\lfoot{\mycourse}
\cfoot{\thepage}
% Creative Commons license BY
% http://creativecommons.org/licenses/?lang=de
\rfoot{\ccby\hspace{2mm}\myauthor}
\renewcommand{\headrulewidth}{0.4pt}
\renewcommand{\footrulewidth}{0.4pt}

\begin{document}
\parindent 0pt
\parskip 6pt

\pagenumbering{Roman} 
\input{special/title}

\clearpage
\thispagestyle{empty}
\tableofcontents

\newpage
\pagenumbering{arabic}
\pagestyle{fancy}

%\vspace{-0.5cm}
\section{Einführung}
Diese Übung soll zur Vertiefung der Begriffe "Authentifizierung und Autorisierung" dienen.
\subsection{Ziele}
Das Ziel dieser Übung ist die Funktionsweise eines Verzeichnisdienstes zu verstehen und Erfahrungen mit der Administration auszuprobieren. Ebenso soll die Verwendung des Dienstes aus einer Anwendung heraus mit Hilfe der JNDI geübt werden. \newline
Authentifizierung bedeutet hier, dass per Username und Passwort eine Anmeldung beim Verzeichnisdienst erfolgt. Autorisierung wird hier im Zusammenhang mit Service-Gruppen und zugeordneten Usern durchgeführt.
\subsection{Voraussetzungen}
\begin{itemize}
	\item Grundlagen Verzeichnisdienst
	\item Administration eines LDAP Dienstes
	\item Verwendung von Commandline Werkzeugen fuer LDAP (LDAPSEARCH, LDAPMODIFY)
	\item Grundlagen der JNDI API für eine JAVA Implementierung
	\item Verwendung einer virtuellen Instanz für den Betrieb des Verzeichnisdienstes
\end{itemize}
\subsection{Aufgabenstellung}
Mit Hilfe der zur Verfuegung gestellten VM wird ein vorkonfiguriertes LDAP Service zur Verfuegung gestellt. Dieser Verzeichnisdienst soll um folgende Eintraege erweitert werden. Das verwendete Namensschema (eg. group.service1 oder vorname.nachname) soll fuer alle Eintraege verwendet werden.
\begin{itemize}
	\item 5 Posix Groups (beliebe Zuweisung von UserIDs)
	\item 10 User Accounts
\end{itemize}
Weiters soll eine Java-Applikationen zur Authentifizierung und Autorisierung entwickelt werden. Folgende Fragestellungen stehen dabei im Mittelpunkt:
\begin{itemize}
	\item Sind Username und Passwort korrekt? 
(Identifikation des Benutzers)
	\item Ist der User berechtigt ein bestimmtes Service zu nutzen?
(Benutzer-Berechtigung)
\end{itemize}
\newpage

\section{Dokumentation der Arbeitsschritte}
\subsection{Grundkonfiguration}
Folgendes Textfile von Prof. Micheler beschreibt das Aufsetzen eines OpenLDAP-Servers, die Grundkonfiguration, Grundlagen in LDAPSEARCH sowie LDAPMODIFY und listet einige weiterführende Links auf. \newline
\begin{lstlisting}[frame=single, caption=Grundkonfiguration]
Installation LDAP:
------------------
sudo apt-get update
sudo apt-get install slapd ldap-utils

sudo dpkg-reconfigure slapd
> DNS domain name: nodomain.com
> Organization name: nodomain
> Administrator password: user
> Database backend: hdb

Installation phpLDAPadmin:
--------------------------
sudo apt-get install phpldapadmin

Configuration phpLDAPadmin:
---------------------------
sudo gedit /etc/phpldapadmin/config.php

$servers->setValue('server','host','localhost');
$servers->setValue('server','base',array('dc=nodomain,dc=com'));
$servers->setValue('login','bind_id','cn=admin,dc=nodomain,dc=com');
$config->custom->appearance['hide_template_warning'] = true;

SSL configuration not performed!

Configuration Apache:
---------------------
/etc/apache2/mods-enabled/alias.conf: following line added
Alias /ldap /usr/share/phpldapadmin/htdocs

Link to phpLDAPadmin:
---------------------
http://localhost/ldap

Modify LDAP Directory:
----------------------
Add new Posix Group: group.default
Add new Posix Group: group.service1
Add new Generic User Account: max.mustermann
Add max.mustermann to group.service1

LDAPSEARCH Commandline Tool / Local:
------------------------------------
ldapsearch -h 127.0.0.1 -p 389 -D "cn=admin,dc=nodomain,dc=com" -W
ldapsearch -h 127.0.0.1 -p 389 -D "cn=max.mustermann,dc=nodomain,dc=com" -W

LDAPSEARCH Commandline Tool / Remote:
-------------------------------------
ldapsearch -h 192.168.0.8 -p 389 -D "cn=admin,dc=nodomain,dc=com" -W
ldapsearch -h 192.168.0.8 -p 389 -D "cn=max.mustermann,dc=nodomain,dc=com" -W -b "dc=nodomain,dc=com"
ldapsearch -h 192.168.0.8 -p 389 -D "cn=max.mustermann,dc=nodomain,dc=com" -W -b "cn=group.service2,dc=nodomain,dc=com" memberUid
ldapsearch -h 192.168.0.8 -p 389 -D "cn=max.mustermann,dc=nodomain,dc=com" -W -b "dc=nodomain,dc=com" "cn=group.*" memberUid
ldapsearch -h 192.168.0.8 -p 389 -D "cn=max.mustermann,dc=nodomain,dc=com" -W -b "dc=nodomain,dc=com" "(objectclass=PosixGroup)"

\end{lstlisting}
\newpage
\begin{lstlisting}[frame=single, caption=Grundkonfiguration]
LDAPMODIFY Commandline Tool / Remote:
-------------------------------------
ldapmodify -h 192.168.0.8 -p 389 -D "cn=admin,dc=nodomain,dc=com" -W  
dn: cn=group.service1,dc=nodomain,dc=com             
changetype: modify
replace: description
description: test


Links:
------
http://docs.oracle.com/javase/tutorial/jndi/index.html
http://www.stefan-seelmann.de/media/presentations/JUGM2008_JavaUndLDAP.pdf
https://www.digitalocean.com/community/tutorials/how-to-install-and-configure-openldap-and-phpldapadmin-on-an-ubuntu-14-04-server
\end{lstlisting}

\subsection{Anlegen von 5 Gruppen und 10 User-Accounts}
In der beschriebenen Testumgebung ist die phpLDAPadmin-Oberfläche via \textit{http://localhost/ldap/} aufzurufen. Im Login-Screen meldet man sich per Credentials \\ \textit{cn=admin,dc=nodomain,dc=com} und \textit{user} an. In der Adminoberfläche findet sich im linken Menü der Eintrag 'Create new entry here'. Eine Liste an Templates für den Erstellungsprozess wird angezeigt, wir wählen zuerst 'Generic: Posix Group' aus. Die GID-Nummer wird automatisch generiert, der Gruppe kann auch ein Name gegeben werden, in unserem Fall 'service1' bis 'service5'. \\ \\
Um einen User zu erstellen findet sich unter 'Create new entry here' der Eintrag 'Generic: User Account'. Relevant bei der Eingabe ist der Vor- und Nachname, die GID-Nummer (zugehörige Gruppe) und das Passwort, der Rest wird automatisch generiert aus den bereitgestellten Daten, kann gegebenenfalls trotzdem noch angepasst werden. \\
\begin{figure}[h]
	\centering \includegraphics[keepaspectratio=true, scale=0.65]{images/10user5groups}
	\caption{10 User-Accounts, 5 Posix-Groups}
\end{figure} \\
\newpage
Hallo!
\newpage

\bibliographystyle{unsrt}
\bibliography{quellen}
\lstlistoflistings
\listoffigures

\end{document}
